\documentclass[12pt]{article}
\usepackage[utf8]{inputenc}
\usepackage{listings}
\usepackage{xcolor}

\definecolor{background}{rgb}{0.95, 0.95, 0.95}
\definecolor{sigWords}{rgb}{250,0,0}
\lstdefinestyle{codeStyle}{
backgroundcolor=\color{background},
keywordstyle=\color{sigWords},
numbers=left,
}

\author{Denitsa Radeva}
\title{report}
\date{March 2021}



\begin{document}


\begin{titlepage}
   \begin{center}
       \vspace*{1cm}

       \textbf{\huge UNIX Coursework}

       \vspace{0.5cm}
        Data Management (1204)
            
       \vspace{1.5cm}

       \textbf{Denitsa Radeva\\Student ID: 31693849}
   \end{center}
\end{titlepage}

\section{My Script}
\lstset{style=codeStyle}
\begin{lstlisting}[showstringspaces=false, language=sh, caption=Extracting information from files.]
#!/bin/bash

for file in $1/*;
do
filename=$(basename $file);
printf $filename" "|sed 's/.dat//g';
cat $file |grep 'Content'|sed 's/>/ /g'
|awk 'BEGIN{count=0}($1=="<Content"){count+=1}
END{print count}';done|sort -k2,2gr
\end{lstlisting}
\section{Description}
The above script is extracting the number of reviews that are mentioned in all data files from a specified directory, which is given as a user input. The name of the script is 'countreviews.sh' and it expects a name or an absolute path of a directory as an argument. The following methodology is used:
\begin{enumerate}
	\item \textbf{For cycle} to loop through all files within a given directory.
	\item The \textbf{basename} utility to extract the current file's name.
	\item Usage of \textbf{sed} to remove the suffix '.dat' from the current file's name.
	\item The utility \textbf{grep} to find all occurrences of the word "Content", because each time it is found, that's exactly $+$1 review more.
	\item The utility \textbf{sed} to replace the symbol '$>$' with a white space in order to separate "$<$Content" from the text of the review and to spot it easily in the next step.
	\item The utility \textbf{awk} to count how many times "$<$Content" 
	is found throughout the document.
	\item Usage of \textbf{sort} in the end to sort in a descending order by the second column of the output, which contains the number of reviews found. 
\end{enumerate}

\end{document}
